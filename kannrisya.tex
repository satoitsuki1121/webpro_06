\documentclass[uplatex,dvipdfmx,a4paper,11pt]{jlreq}
\usepackage[utf8]{inputenc}
\usepackage{url}
\usepackage{geometry}
\usepackage{enumerate}
\usepackage{listings}

% プログラムコード表示用の設定
\lstset{
    basicstyle=\ttfamily\small,
    frame=single,
    breaklines=true,
}

\geometry{left=25mm,right=25mm,top=30mm,bottom=30mm}

\title{Webアプリケーションシリーズ 管理者向け仕様書\\ \large --- システム構築・運用管理マニュアル ---}
\author{25G1058 佐藤樹}
\date{\today}

\begin{document}

\maketitle

\section{システム概要}
本システムは,Node.jsおよびExpressフレームワークを用いて構築された,3つの情報一覧サービス(F1チーム情報,歴代GT-R情報,ディズニー周年パレード情報)を統合管理するWebアプリケーションシステムです .

管理者は,1つのサーバープロセスを起動するだけで,これらすべてのサービスを同時に一般利用者へ提供することが可能です .

\section{動作環境と構成}

\subsection{必要ソフトウェア}
本システムを立ち上げるには,以下の環境が必要です.
\begin{itemize}
    \item \textbf{Node.js}: 実行エンジン
    \item \textbf{npm}: パッケージ管理ツール
    \item \textbf{Express}: Webアプリケーションフレームワーク 
    \item \textbf{EJS}: テンプレートエンジン 
\end{itemize}

\subsection{ファイル構成}
システムを構成する主なファイルは以下の通りです.
\begin{itemize}
    \item \texttt{app5.js}: サーバー本体(各サービスのルート定義とデータ管理) 
    \item \texttt{views/F1.ejs}: F1一覧表示用テンプレート [cite: 1]
    \item \texttt{views/GT.ejs}: GT-R一覧表示用テンプレート [cite: 9]
    \item \texttt{views/TD.ejs}: パレード一覧表示用テンプレート [cite: 5]
    \item \texttt{public/}: 画像ファイルなどの静的資産を格納するディレクトリ 
\end{itemize}

\section{システムのセットアップ}

\subsection{依存ライブラリのインストール}
サーバー上でターミナルを起動し,以下のコマンドを実行して必要なライブラリを準備します.
\begin{lstlisting}
npm install express ejs
\end{lstlisting}

\subsection{サーバーの起動}
以下のコマンドを実行し,サーバーを立ち上げます.
\begin{lstlisting}
node app5.js
\end{lstlisting}
正常に起動すると,ターミナルに「Example app listening on port 8080!」と表示されます .管理者は \url{http://localhost:8080/} を通じて各サービスへアクセスできます .

\section{データの管理}

\subsection{初期データの設定}
各サービスのデータは \texttt{app5.js} 内の変数(配列)として管理されています.
\begin{itemize}
    \item \textbf{F1チームデータ}: 変数 \texttt{team} 
    \item \textbf{GT-Rデータ}: 変数 \texttt{car} 
    \item \textbf{パレードデータ}: 変数 \texttt{pare} 
\end{itemize}

\subsection{実行中のデータ追加}
管理者は,専用の追加用URL(エンドポイント)を介して,ブラウザから動的にデータを追加することができます .
\begin{itemize}
    \item \textbf{F1}: \texttt{/F1\_add} 
    \item \textbf{GT-R}: \texttt{/GT\_add} 
    \item \textbf{パレード}: \texttt{/TD\_add} 
\end{itemize}
追加されたデータは,サーバーが稼働している間,各変数内に保持されます .

\section{保守・トラブルシューティング}
\begin{itemize}
    \item \textbf{画像の追加}: 新しい画像を登録する際は,\texttt{public} ディレクトリ内にファイルを配置し,データ登録時の \texttt{image} パラメータにパス(例: \texttt{public/new\_image.jpg})を指定してください .
    \item \textbf{データの永続性}: 現在の仕様では,データはメモリ上の変数に保存されています .サーバーを再起動すると,実行中に追加されたデータはリセットされ,プログラム内の初期値に戻る点に注意してください.
    \item \textbf{厳格モード}: プログラムの品質維持のため,すべてのJSファイルの先頭に \texttt{"use strict";} を記述して管理してください.
\end{itemize}

\end{document}