\documentclass[uplatex,dvipdfmx,a4paper,11pt]{jlreq} % 日本語に適した設定に変更
\usepackage[utf8]{inputenc}
\usepackage{url}
\usepackage{geometry}
\usepackage{enumerate}

% ページの余白設定
\geometry{left=25mm,right=25mm,top=30mm,bottom=30mm}

\title{Webアプリケーションシリーズ 利用者向け仕様書\\ \large --- F1・歴代GT-R・周年パレード案内 ---}
\author{25G1058 佐藤樹}
\date{\today}

\begin{document}

\maketitle

\section{はじめに}
本システムは,特定のテーマ(F1チーム,日産GT-R,ディズニーランドのパレード)に関する情報を一覧で確認したり,新しい情報を追加したりすることができるWebアプリケーションです.パソコンやスマートフォンのブラウザから,どなたでも簡単に操作することができます.

\section{提供する3つのサービス}
本システムでは,以下の3つの情報を閲覧・追加できます.

\begin{enumerate}[1.]
    \item \textbf{F1 2025 コンストラクター順位表} \\
    最新のF1チームの順位,チーム名,チームロゴなどの情報を表示します.
    \item \textbf{歴代 GT-R ギャラリー} \\
    初代「ハコスカ」から最新のR35型,コンセプトカーまでの歴代モデルを表示します.
    \item \textbf{ディズニーランド 周年パレード年表} \\
    東京ディズニーランドの開園5周年から現在までの,歴代アニバーサリーパレードを表示します.
\end{enumerate}

\section{基本的な使いかた}

\subsection{一覧画面を見る}
ブラウザのURL入力欄に,以下の目的のURLを入力して「Enter」キーを押してください.

\begin{itemize}
    \item \textbf{F1の情報を知りたい場合}: \url{http://localhost:8080/F1}
    \item \textbf{GT-Rの情報を知りたい場合}: \url{http://localhost:8080/GT}
    \item \textbf{パレードの情報を知りたい場合}: \url{http://localhost:8080/TD}
\end{itemize}

画面には,写真,名称,関連するコード(ポイントや通称など)が表形式で分かりやすく表示されます.

\subsection{新しいデータを追加する}
情報を新しく登録したい場合は,URLの末尾に情報を書き加えることで追加が可能です.

\paragraph{操作例(F1チームを追加する場合)}
URLの後に,以下のルールで文字を付け加えます.
\begin{quote}
    \texttt{/F1\_add?id=11\&code=50\&name=新チーム\&image=public/image.jpg}
\end{quote}
これを入力して実行すると,自動的に一覧画面へ戻り,新しい情報が一番下に追加されます.

\section{困ったときは}
\begin{itemize}
    \item \textbf{画像が表示されない}: インターネットの接続状況を確認するか,ページを再読み込みしてください.
    \item \textbf{画面が真っ白になる}: 入力したURLが間違っている可能性があります.つづり(スペル)を再度ご確認ください.
\end{itemize}

\section{ソースコードの公開場所}
本システムの設計図(ソースコード)は,以下のGitHubリポジトリにて公開されています.

\begin{center}
    \url{https://github.com/satoitsuki1121/webpro_06.git}
\end{center}

\end{document}