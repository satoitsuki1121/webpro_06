\documentclass[uplatex,dvipdfmx,a4paper,11pt]{jlreq}
\usepackage[utf8]{inputenc}
\usepackage{url}
\usepackage{geometry}
\usepackage{enumerate}
\usepackage{listings}
\usepackage{color}

\lstset{
    language=JavaScript,
    basicstyle=\ttfamily\small,
    keywordstyle=\color{blue},
    stringstyle=\color{red},
    commentstyle=\color{green},
    frame=single,
    breaklines=true,
}

\geometry{left=25mm,right=25mm,top=30mm,bottom=30mm}
\url{https://github.com/satoitsuki1121/webpro_06.git}

\title{開発者向け仕様書:歴代GT-Rシステム}
\author{25G1058 佐藤 樹}
\date{\today}

\begin{document}

\maketitle

\section{システム概要}
本プログラムは,日産GT-Rの歴代モデル情報を管理・表示するためのWebアプリケーションです.Expressによるサーバサイド処理とEJSテンプレートを組み合わせ,データの動的な表示と追加機能を実現しています.

\section{データ構造}
データはサーバ側の変数 \texttt{car}(オブジェクト配列)に格納されます.
\begin{itemize}
    \item \textbf{id}: 登場順を示す数値.
    \item \textbf{code}: 車両の通称(「ハコスカ」等)を格納する文字列.
    \item \textbf{name}: 正式な車種名を示す文字列.
    \item \textbf{image}: 画像ファイルへのパス(\texttt{public/}配下).
\end{itemize}

\section{リソースと機能詳細}
本システムは,以下のHTTPメソッドとリソース名を用いて操作を定義しています.
\begin{table}[h]
\centering
\begin{tabular}{|l|l|l|l|}
\hline
機能 & HTTPメソッド & リソース名 & 処理内容 \\ \hline
一覧表示 & GET & \texttt{/GT} & 全データの抽出と一覧表示 \\ \hline
データ追加 & GET & \texttt{/GT\_add} & クエリパラメータを用いた新規登録 \\ \hline
データ削除 & DELETE (予定) & \texttt{/GT/:id} & 指定IDのデータ削除 \\ \hline
データ編集 & PUT (予定) & \texttt{/GT/:id} & 指定IDのデータ更新 \\ \hline
\end{tabular}
\end{table}

\section{ページ遷移と制御フロー}
\subsection{遷移の仕組みとユーザー導線}
利用者が \url{http://localhost:8080/GT} にアクセスすると一覧画面が描画されます.データの追加は \texttt{/GT\_add} を通じて行われます.車種名(または通称)をクリックするなどのアクションにより,詳細情報の取得や画面遷移が発生する構造となっています.
\subsection{操作後の表示内容}
追加・変更・削除の各操作が完了した後は,サーバ側で一覧画面へ強制的に遷移させます.これにより,利用者は常に最新のデータ状態を確認することが可能です.

\section{授業で説明していない技術の概要と採用理由}
本システムの開発にあたり,講義資料の範囲外で採用した技術について以下に記します.

\begin{enumerate}[1.]
    \item \textbf{REST API(設計思想)}
    \begin{itemize}
        \item \textbf{概要}: URLを「リソース(物)」として扱い,HTTPメソッドで操作を決定する設計手法です.
        \item \textbf{採用理由}: 統一されたルールでURLを設計することで,プログラムの拡張性と可読性を高めるため採用しました.
    \end{itemize}
    
    \item \textbf{res.redirect()(リダイレクト機能)}
    \begin{itemize}
        \item \textbf{概要}: 処理完了後,サーバからブラウザに対して別のURLへ再アクセスするように指示を出す機能です.
        \item \textbf{採用理由}: データ追加後にスムーズに一覧画面へ戻るユーザー体験を提供し,二重登録などの誤作動を防ぐため採用しました.
    \end{itemize}

    \item \textbf{express.static(静的ファイル配信設定)}
    \begin{itemize}
        \item \textbf{概要}: サーバ上の特定のフォルダ内のファイルを,外部から直接参照可能にするミドルウェアです.
        \item \textbf{採用理由}: 車両画像などの画像資産をEJSテンプレートから効率的に呼び出し,安定して表示させるために採用しました.
    \end{itemize}

    \item \textbf{"use strict";(厳格モード)}
    \begin{itemize}
        \item \textbf{概要}: JavaScriptの記述ミスをエラーとして検出しやすくする宣言です.
        \item \textbf{採用理由}: プログラムの動作を安定させ,バグの混入を未然に防ぐ品質管理のために採用しました.
    \end{itemize}
\end{enumerate}


\end{document}